%%%%%%%%%%%%%%%%%%%%%%%%%%%%%%%%%%%%%%%%%%%%%%%%%%%%%%%%%%%%%%%%%%%%%%%%%
%%
%W  cyclic.tex           GrpConst documentation             Bettina Eick
%%

%%%%%%%%%%%%%%%%%%%%%%%%%%%%%%%%%%%%%%%%%%%%%%%%%%%%%%%%%%%%%%%%%%%%%%%%%
\Chapter{The Cyclic Split Extension Method}

\index{The Cyclic Split Extension Method}

This is a method to construct up to isomorphism the groups of order
$p^n \cdot q$ for different primes $p$ and $q$ which have a normal
Sylow subgroup. We first describe the main function for this method
and then functions for a slightly more low level access to the
algorithms. 

Note that all functions described in this chapter rely on an
efficient method for `AutomorphismGroup' for $p$-groups. Such
a method is provided in the package AutPGrp.
Thus it is useful to install and load this share package before
using the functions described in this chapter.

%%%%%%%%%%%%%%%%%%%%%%%%%%%%%%%%%%%%%%%%%%%%%%%%%%%%%%%%%%%%%%%%%%%%%%%%%
\Section{The Main Function}

\> CyclicSplitExtensionMethod( <p>, <n>, <q> ) F
\> CyclicSplitExtensionMethod( <p>, <n>, <q>, <uncoded> ) F

Clearly, each of the computed groups is a split extension of a group
of order $p^n$ and the cyclic group of order $q$. The output is a
record with three entries <up>, <down> and <both>. Each of these 
contains a list of groups, <both> the nilpotent groups, <up> the 
remaining groups with a normal Sylow $p$-subgroup and <down> the
remaining groups with normal Sylow $q$-subgroup.

As in Chapter "The Frattini Extension Method" all groups are
described as codes. Setting <uncoded> to true, the function
will return pc groups instead.

If one wants to construct the groups of order $p^n \cdot q$
for fixed $p$ and several primes $q$, it is more efficient to
do this in one go. Thus it is possible to hand a list of primes
for the input <q>.

\beginexample
gap> CyclicSplitExtensionMethod( 2,2,7, true );    
rec( up := [  ], 
  down  := [ <pc group of size 28 with 3 generators>, 
             <pc group of size 28 with 3 generators> ], 
  both  := [ <pc group of size 28 with 3 generators>, 
             <pc group of size 28 with 3 generators> ] )

gap> CyclicSplitExtensionMethod( 2,2,[3,5], true );
rec( up := [ <pc group of size 12 with 3 generators> ], 
  down  := [ <pc group of size 12 with 3 generators>, 
             <pc group of size 20 with 3 generators>, 
             <pc group of size 20 with 3 generators>, 
             <pc group of size 12 with 3 generators>, 
             <pc group of size 20 with 3 generators> ], 
  both  := [ <pc group of size 12 with 3 generators>, 
             <pc group of size 20 with 3 generators>, 
             <pc group of size 12 with 3 generators>, 
             <pc group of size 20 with 3 generators> ] )
\endexample

Note that the function `CyclicSplitExtensionMethod' requires that
the groups of order $p^n$ are given within the SmallGroups Library. 

%%%%%%%%%%%%%%%%%%%%%%%%%%%%%%%%%%%%%%%%%%%%%%%%%%%%%%%%%%%%%%%%%%%%%%%%%
\Section{The Underlying Functions}

It is possible to construct the cyclic extensions of a single group
of order $p^n$ only. The output is as above.

\>CyclicSplitExtensions( <G>, <q> ) F
\>CyclicSplitExtensions( <G>, <q>, <uncoded> ) F

Moreover, the computation of the record entry <up> and the record
entry <down> can be separated by using the following functions.

\> CyclicSplitExtensionsUp( <G>, <q> ) F
\> CyclicSplitExtensionsUp( <G>, <q>, <uncoded> ) F

\> CyclicSplitExtensionsDown( <G>, <q> ) F
\> CyclicSplitExtensionsDown( <G>, <q>, <uncoded> ) F

The input for these functions is the same as above. The first
function returns a list of groups with one normal subgroup of order
$p^n$ and the second a list of groups with one normal subgroup of order
$q$. 

\beginexample
gap> G := SmallGroup( 16, 10 );;
gap> CyclicSplitExtensionsUp( G, 3, true );
[ <pc group with 5 generators> ]

gap> G := SylowSubgroup( SymmetricGroup(4), 2);
Group([ (1,2), (3,4), (1,3)(2,4) ])
gap> CyclicSplitExtensionsDown( G, 3 );
[ rec( code := 6562689, order := 24 ), 
  rec( code := 2837724033, order := 24 ) ]
\endexample
